\chapter{Premier chapitre}
\label{chap:premierchapitre_exemple}

\section{Une section}
Lorem ipsum dolor sit amet, consectetur adipiscing elit \cite{Roque2012,Roque2012b,Roque2012c,Roque2012d}. 
Curabitur eu amet (fig. \ref{fig:une-image}). Deux citations \cite{Arapoglou2011,Roque2013c}.

\begin{figure}[htp]
  \centering
  \input{images/tikz_diagram}
  \caption{Exemple de diagramme TikZ.}
  \label{fig:une-image}
\end{figure}

\section{Une autre section}
 bla bla bla \cite{Thompson2013raey}.


\subsection{Código fonte}
A inserção de código fonte deve ser por meio do comando \textit{lstlisting}.

\begin{lstlisting}
int main(){
  int a,b,c;
  float x;
  printf("informe o tamanho do lado do quadrado");
  scanf("%d", &a);
  printf("A area do quadrado %d", b=area(a));
  printf("Duas vezes o valor do lado do quadrado %d", c=aumenta(a));
}
\end{lstlisting}

%%% Local Variables: 
%%% mode: latex
%%% TeX-master: "isae-report-template"
%%% End: 

\chapter{Un chapitre}
\label{sec:unchapitre}


 Encore une citation \cite{Cadambe2008}.

\begin{figure}[htp!]
  \centering
  \setlength\figureheight{7cm}
  \setlength\figurewidth{9cm}
  \input{images/tikz_plot}
  \caption{Exemple de courbe TikZ.}
  \label{fig:courbe-tikz}
\end{figure}

\section{Analyse aux limites}


\subsection{Quelques détails sur cette méthode}


\subsection{On n'est jamais très fort pour ce calcul}


\begin{align}
H_{m,n,p,q} &= \DPR{\rproto_{p,q}}{\OP{H} \tproto_{m,n}}\\
&= \iint\limits_{\SET{R}^2} S_{\OP{H}}(f,\tau) \DPR{\rproto_{p,q}}{\OP{U}_{f,\tau} \tproto_{m,n}} \ud f \ud \tau.
\end{align}

\section{Vérification par simulation numérique}


%%% Local Variables: 
%%% mode: latex
%%% TeX-master: "isae-report-template"
%%% End: 