\chapter*{Introduction}
\addcontentsline{toc}{chapter}{Introduction}
\markboth{Introduction}{Introduction}
\label{chap:introduction}
%\minitoc

Dans le cadre de la formation au département d'Électronique et Traitement du Signal à l'ENSEEIHT, école d'ingénieur de l'INP de Toulouse.

Électronique é top.


\section*{Présentation du projet}

L'UCM Nano est une unité de contrôle électronique utilisée dans des véhicules industriels. Ce contrôleur est un composant complexe qui nécessite un ou plus microprocesseurs pour opérer le véhicule.

% Ajouter du texte sur le projet

Cet unité de contrôle utilise les protocoles de communication suivants : bus de communication CAN, portes de protocoles de communication réseau en série LIN et protocoles de réseau local à commutation de paquets Ethernet. Elle possède des entrées analogiques, entrées de fréquence, sorties numériques et PWM (LSD et HSD) et des sorties spéciaux. La quantité de chaque élément est la suivante :

% A Controller Area Network (CAN bus) is a robust vehicle bus standard designed to allow microcontrollers and devices to communicate with each other in applications without a host computer.

% LIN (Local Interconnect Network) is a serial network protocol used for communication between components in vehicles. 

% Ethernet, ou Éthernet1,2,3, est un protocole de réseau local à commutation de paquets. C'est une norme internationale : ISO/IEC 8802-3.

\renewcommand{\labelitemi}{\textbullet}
\begin{itemize}
    \item 2 bus CAN ;
    \item 2 portes LIN ;
    \item 2 connecteurs Ethernet ;
    \item 20 entrées analogiques ;
    \item 4 entrées de fréquence ;
    \item 33 sorties numériques (12 LSD et 11 HSD) ;
    \item 5 sorties spéciaux.
\end{itemize}

